If you are reading this on the Christmas Eve, then my plan worked out perfectly. If not, then please accept my apologies, because something beyond my control messed up. But for now I will assume that everything went fine.

This small book represents the 4 months I spend in Seoul on my exchange at KAIST Business School. The content is based on my blog S(e)oul eXchange that I managed to keep alive and running for the whole 4 months I was there. I really appreciated that some people were actually reading the blog on a regular basis, as it is easier to write for some audience. I hope that my posts were interesting for you and that I succeeded in capturing my experiences and thoughts about Korea. Unfortunately, as always, some of the experiences were simply impossible to record with words and blogposts, so unless you manage to get into my brain, they will stay only as memories forever.

The idea to create this book came from my sister, that told me during one Skype conversation \textit{``I really like what you are writing, you are the awesomest brother in the world, you should publish it and make tons of money''}. Or something along that lines. When my little nephew supported it with his \textit{``Gaga onga bugubu''}, I realized that it's not a bad idea, and that it would actually make a nice Christmas present.

I do not want to bother you with technical details, so just note that I used Python to get the posts from the blog and \LaTeX for typesetting. And honestly, I really enjoyed doing it, as it brought me back to my geeky self, that likes to tinker with scripts and typesetting and fonts etc. I started compiling the book sometime in October and finished it in December, when I wrote this introduction and the closing part, did the last clean-ups and fixes and sent the file to the printshop. It was really nice to go through the book, read parts of the posts and remember all the nice moments. Well, see for yourself...

Merry Christmas

\begin{figure}[h]
\raggedleft\includegraphics[height=2\baselineskip]{photos/signature}
\end{figure}
