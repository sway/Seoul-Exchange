\begin{post}
	\postdata{Culinary Adventures II.}{2011}{11}{15}{22}{57}{7}
	\begin{content}
<blockquote>Me: "So, how is the food in Korea?"
Roman: "Well, do you like spicy food?"
Me: "Yeah, kind of."
Roman: "They you'll like it..."</blockquote>
This was my conversation with my friend Roman few days before coming to Korea. After more than two months here, my response to his question would probably be different. It's not that I don't like spicy food anymore, I just don't like the Korean kind of spicy. And who or what is to blame? The cafeteria! As I realize it now, the only spicy food that I did not like was from the KAIST cafeteria (and the curry place, but that was probably caused by my hangover). It does not matter if it was the "weird meat in a brown sauce" (for my Czech readers, it is like some kind of UHO) or the "kimchi rice", it was just too spicy and tasteless. I know that you might say "you have to get used to it" or "you are a sissy" or even "OMG WTF, you are lame and you suck, ROFL", but I just can't help it. I like food that tastes good already the first time, not after 3 months of "getting used to", especially because there are so many other opportunities to eat good food.

<em>(The fun part of this post ends here...but there is still some interesting info down there, so just carry on reading...)</em>

Last time I promised to walk you through some of the places we go to. The problem is, that I find such descriptive writing quite boring. So I will try to keep it as short as possible...I hope you don't mind...
<h1>Fast foods</h1>
What would be do without good old fastfood chains, right. In the immediate vicinity to our campus, there are several branches of both Korean and foreign fast food chains.

The most famous one is definitely <strong>Burger King</strong>. The only special thing about Korean BK's is the Bulgogi Burger, a local specialty with beef and a special bulgogi sauce. Honestly, I prefer the classical Whopper and based on my observations, so do Koreans. As opposed to other FF chains, most BK's in Korea are open 24/7.

Another burger place is a Korean chain called <strong>Young-Cheol Burger</strong>. They do not serve classical burgers, but most of their so called burgers are more like Subway subs with ham and vegetables. I really appreciate that every sub is freshly made, so you are sure it has not been sitting on the shelf for already 20 minutes.

Moving on...<strong>Isaac Toast</strong>. A small but nice place, where they sell toasts. Since their menu is only in Korean and I haven't bothered with translation, I do not know what toasts they have, except for a M.V.P. toast, which has a beef patty, egg, raddish, cheese and three sauces. The important thing is that it's good, relatively cheap and it certainly fills you up.

And now to the sweet stuff. Right next to our regular place there is <strong>Dunkin' Donuts</strong>, where we usually buy our dessert after eating at the regular place. Across the street from it there is <strong>Baskin Robbins</strong> (ice cream), and a little further up the street there is <strong>Starbucks</strong>, surrounded by dozen other coffeeshop, that are bigger, cheaper and not-at-all ripping off the image of SBs with a round logo and a bold, sans-serif font.
<h1>Tonkatsu!</h1>
Tonkatsu with cheese, rice and salad. A little bowl of miso soup on the side. Yum! The Japanese fried pork cutlet (tonkatsu) is suprisingly very popular in Korea, any you can find it in almost any restaurant that does not specialize on some specific cuisine. There are two places that we sometimes go to for tonkatsu — one serves the tonkatsu as described above, as well as some udong noodles, sushi and other Japanese things, while the other one is more inclined towards Italian cuisine, and serves the cutlet with mozzarella and bolognese saucejifydshfnfbchvgwhfkc.

<em>(And this is the place where I stopped having fun writing this and my head fell on my keyboard...)</em>

Sometimes I feel the obligation to write informative stuff to educate people and tell them something about Korean culture. It also helps me to better sort out my thoughts and experiences. As you might have noticed, I do like to write stuff. But it has to be something at least remotely interesting for me and for you. And frankly, <strong>I don't think anyone is interested in this phonebook of places to eat anymore.</strong>

<strong>You know, food is quite a difficult thing to write about. It is awesome to eat, in most cases nice to look at and smell, for me also fun to cook, but it is quite boring to write about, especially when I need to remember all the places and food and look it up on Wikipedia to describe it properly. So if you don't mind, I will try to summarize my ad interim Culinary Adventures in one short paragraph.</strong>

Korean cuisine is awesome. Bulgogi, Kimbap, Mandu, Galbi, Kimchi, Ramen, Bibimbap, Udong, yellow raddish, white raddish, chopsticks, meat on the stick, everything. Apart from the cafeteria and a pork intestine place, everything has been at least acceptable, but in most cases good or delicious. If you ever come to Korea, go for it and try the local meals. You don't have to worry, even though they do eat dogs and live octopus, it's not really common. So go ahead, take your flat metal chopsticks and enjoy your meal. And remember, rice is eaten with a spoon and you should never stick your chopsticks into the ricebowl.

<em>N.B.: Sorry for this messy post...I really wanted to write about the food here, but I guess I just took it from the wrong side. I'll try to sum everything up at the end, so if you now think I am merely crazy, don't judge me and wait for December. If you don't change your opinion by then, then I probably really am crazy.</em>
	\end{content}
\end{post}
