\begin{post}
	\postdata{Fieeeeeeld triiiiiip!}{2011}{11}{12}{22}{00}{47}
	\begin{content}
After two months of Korean Business and Culture lectures we finally got to do our first field trip. The destination was a Korean corporation (not as big as a \textit{chaebol}) Kolmar, that specialized on cosmetics and pharma products. We have visited their Sinjeong factory, which comprises a pharma factory, a pharma R\&D lab and a skin care factory.

To educate you a little, let me give you a quick review of the history of the company. Compared to big Korean corporations, Kolmar is quite young --- it was founded only 21 years ago in 1990. Its parent company was Kolmar Americas Inc., that manages the global network of Kolmar Global. The Korean branch was established as a joint-venture with Kolmar Japan.

At first, Kolmar Korea focused mainly on cosmetics. One of their strategy was ``perfection'', which lead to CGMP (Cosmetic Good Manufacturing Practice) certification in 1994. Until 2000, the company continued growing and received several awards and prizes for their results and development. In 2001 they were the first ones to receive the ISO9001 certification (Quality Management certification) in Korea. Year 2002 marked the beginning of the pharmaceutical business, as well as KGMP certification of all facilities. They were also listed at the KRX (Korea Exchange) in the same year. In the following years, they have enlarged the portfolio of manufactured products and strengthened their R\&D capabilities by extending the R\&D labs and centers. In 2007, a Chinese spin-off, Kolmar Beijing, was established to simplify access to the Chinese market. Between 2000 and 2010, Kolmar was awarded several other certifications, such as Ecocert, Lohas or ISO 14001, and in 2011, they received as the first Korean company the ISO 22716 certification for good manufacturing practices in cosmetics.

Currently, the company focuses on subsidiary manufacturing and R\&D for companies such as L'Oreal, Shiseido, P\&G, Unilever, Dior or Johnson-Johnson. They also supply local Korean manufacturers (Nature Republic, Beyond, Tonymoly or The FaceShop). The CEO is {\H 윤동한} (Yoon, Dung-Han), and they have approx. 1000 employees.

The field trip begun in the morning, when we gathered at the main gate of our campus. Since not all exchange students participated in the trip, the bus was half empty, so almost everybody got the double seat for their own. I assume that the ride was nice, but I can't tell you any details, because I immediately fell asleep after departing from KAIST and woke up few minutes before arriving to the factory. As mentioned above, we visited the Sinjeong factory, which is about 1:40h from Seoul.

Upon arrival, we had to take of our shoes and take on some slippers instead, however, since my foot is bigger than average Korean (I have 290, which is usually the biggest size they have in shops), it was not very comfortable.  What was comfortable, however, was the provided ``french coffee'' in the presentation room, where we got a formal welcoming speech and a short introduction about the company.

%<a href=``http://soulexchange.wordpress.com/2011/11/12/fieeeeeeld-triiiiiip/p1010418/'' rel=``attachment wp-att-324''><img class=``aligncenter size-medium wp-image-324'' title=``Fancy lab tools'' src=``http://soulexchange.files.wordpress.com/2011/11/p1010418.jpg?w=530'' alt=``'' width=``530'' height=``263'' /></a>

The first stop on our factory tour was the pharma R\&D lab. I was really surprised, that there were no security measures at all. We could take any photos and our movement was virtually unrestricted. We were walking through the labs, where various liquids were bubbling or swirling, and different screens were showing different graphs or curves. Fancy stuff indeed! I even got a souvenir --- a capsule from a capsule maker. I am not sure what's inside, but it's cool. I will give it to someone as a Christmas present, I guess{\ldots}just kidding, it's all mine!

%<a href=``http://soulexchange.wordpress.com/2011/11/12/fieeeeeeld-triiiiiip/p1010404/'' rel=``attachment wp-att-325''><img class=``aligncenter size-medium wp-image-325'' title=``Capsule making thingie'' src=``http://soulexchange.files.wordpress.com/2011/11/p1010404.jpg?w=530'' alt=``'' width=``530'' height=``353'' /></a>

The second part was a visit to the skin care factory. For that, we had to take on shoe covers, lab coats and hairnets. It is amazing how a group of MBA students can turn into a bunch of kids when they are given a hair net and shoe covers that you can slide on.

%<a href=``http://soulexchange.wordpress.com/2011/11/12/fieeeeeeld-triiiiiip/p1010430/'' rel=``attachment wp-att-326''><img class=``aligncenter size-medium wp-image-326'' title=``A group of weirdos in weird clothes'' src=``http://soulexchange.files.wordpress.com/2011/11/p1010430.jpg?w=530'' alt=``'' width=``530'' height=``261'' /></a>

Our excitement levels got even higher when they told us we have to go through an air shower. 8 people at a time went into a sealed chamber, where a set of air jets got rid of particles that could influence the manufacturing process. Even though we could observe the manufacturing process only through a window, we saw the complete filling and packaging process, as well as the storage are for the creams and stuff before packaging.

%<a href=``http://soulexchange.wordpress.com/2011/11/12/fieeeeeeld-triiiiiip/p1010431/'' rel=``attachment wp-att-327''><img class=``aligncenter size-medium wp-image-327'' title=``Packaging and quality control'' src=``http://soulexchange.files.wordpress.com/2011/11/p1010431.jpg?w=530'' alt=``'' width=``530'' height=``315'' /></a>

%<a href=``http://soulexchange.wordpress.com/2011/11/12/fieeeeeeld-triiiiiip/p1010440/'' rel=``attachment wp-att-329''><img class=``aligncenter'' title=``Barrels full of magic'' src=``http://soulexchange.files.wordpress.com/2011/11/p1010440.jpg?w=530'' alt=``'' width=``530'' height=``198'' /></a>

It is weird to see the big blue barrels full of creams, that are waiting to be packaged into small tubes. I bet many girls would be happy to have access to such barrel. After some more time of sliding through to corridor, we got out through the air shower chamber (no shower this time, though), and went for lunch. The bulgogi was decent, as was the soup and other things.

%<a href=``http://soulexchange.wordpress.com/2011/11/12/fieeeeeeld-triiiiiip/p1010446/'' rel=``attachment wp-att-330''><img class=``aligncenter size-medium wp-image-330'' title=``Bulgogi!'' src=``http://soulexchange.files.wordpress.com/2011/11/p1010446.jpg?w=530'' alt=``'' width=``530'' height=``311'' /></a>

After lunch we had the last part of our field trip --- presentation of Kolmar's business model in Korea. A big part of it was already presented in the ``history'' part, so I will just quickly summarize my impressions. Kolmar largely focuses on the quality of their manufacturing process. Multiple certifications clearly show their determination to comply with all different quality requirements given by both Korean (KFDI) and international agencies, and provide leverage for cooperation with foreign subjects. Currently, they are the only ODM (Original Design Manufacturer) in Korea, which gives means a great potential for growth on both local and foreign market. By not focusing solely on manufacturing, but also in R\&D (in 2009 their R\&D-to-sales ratio of 6\% was about 3\% higher than their competitors') they also profile themselves as an innovative company with active development, which as well increases their leverage for their customers.

In future, it is possible that Kolmar will launch their own brand of cosmetics (not that probable for pharma), in which they would utilize all the experience gained during partnership with some of the biggest cosmetics companies in the world, as well as in Korea. In my opinion, it would make more sense to target the local market, because I feel like Koreans are still very much inclined toward local products than imported ones.

%<a href=``http://soulexchange.wordpress.com/2011/11/12/fieeeeeeld-triiiiiip/p1010444/'' rel=``attachment wp-att-328''><img class=``aligncenter size-medium wp-image-328'' title=``Colorful palettes'' src=``http://soulexchange.files.wordpress.com/2011/11/p1010444.jpg?w=530'' alt=``'' width=``530'' height=``154'' /></a>

This is the end of one of the most boring posts on my blog so far. I am sorry if I bored you to death, but there is a reason for this post --- I want to use it as a foundation for my report for the KBC class, so I figured that it would be easier to first summarize the trip in more informal manner, and then build the report based on the blogpost than vice versa. So stay tuned, there will be more funny posts soon!
	\end{content}
\end{post}
