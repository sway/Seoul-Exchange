\begin{post}
	\postdata{A kindergarten weekend in Hansan}{2011}{12}{5}{1}{32}{33}
	\begin{content}
Boosting your region's reputation by inviting and subsidizing tourist trips might be useful, however, you have to choose your target group wisely. After this weekend I am convinced that we were not the right one.

The Hansan Treasure Hunting tour --- \textit{``A night and a day with starry-starry jewel-like culture and nature of Hansan-myun, Seocheon-gun, Chungcheongnamdo''.} Sounds lovely, right. Well, marketing is one thing, and reality is the other one. The whole trip started on Saturday \sout{morning} night. Seriously, departing at 8am from Gangnam is an insane idea. Despite catching the only traffic jam in Seoul on Saturday morning we managed to get to the bus on time and even snatch some breakfast and coffee. The bus was fairly empty, as the only participants was our KAIST group and two people from KU --- Shafa and Sam.

Hansan is located in the east part of Korea, a little bit under Daejeon, close to Seocheon. It used to be a fairly big city with a big market, however, as people moved to bigger cities, it became just a small and empty city. The initiative that organized the Eco-tour tries to revive the traditional market by attracting tourists into the locality and showing them around. . The trip was highly subsidized by various entities, so we ended up paying only 30,000KRW for the whole weekend.

We started out with a lunch, which was nice. The tofu soup is not my favorite dish of all times, but it is a traditional meal of Koreans, so it was fine. The tour continued to an old pharmacy, galvanizing shop, local market (few stands with fishes and vegetables, one with dog meat), a traditional blacksmith, makkoli factory and a shop with local products.

Next stop was a place where they manufactured the local fabric called (?) (damn, I forgot) . The whole complex looked awesome. A grassy square was surrounded by traditional houses with straw roofs, and it looked somehow clean and peaceful. In the interior we were shown how the individual thread are separated from the bigger braid and how the fabric is weaved on the loom.

The part that I was most exciting about before the trip and most disappointed about after was the following bicycle ride. On rented mountain bikes, which were tiny 24'' ``Tesco-bikes'' we went on a short trip to a reed field nearby. To make sure we are safe they tried to make us wear elbow and knee pads, which we all refused and wore only the helmet. To further ensure our safety we got advised not to use the front brake, because we could injure ourselves by doing so. Ehm, the number next to our age was in years, not months{\ldots}anyway, after approx. 20 minutes of dangerous 10km/h ride we arrived to the ``famous'' reed field, where the movie JSA was shot. Surprisingly, the only thing we found there was reed. Fun fun fun\ldots

For evening we had planned a dinner, ``natural dyeing'' and ``local drinks and snacks''. Personally, I was already dead (or should I say ``dyed'') and hungry, so I really appreciated the bibimbap (one of the best one I had so far). After dinner each of got a piece of white cloth and two rubber rings. Long story short, we tied the rubber rings around the cloth, submerged the whole thing into onion skin broth and kept it there for 20 minutes. However, because we had to rinse it in the bath with our hands, not only the cloth, but also my hand was completely yellow. And it did not want to go down. After that we put it in yet another bath, this time with some color (I chose khaki) and left it there. And then there was nothing{\ldots}void{\ldots}eternal emptiness. 7pm, in the middle of nowhere, no 7-eleven, no bar, not even a weird pub with few patrons. What can you do in such situation{\ldots}Jump in a car and go shopping. Fortunately, Jin managed to persuade the owners to take us ``downtown'', so after 30 minutes we had 18 bottles of beer and 10 bottles of soju, snacks, cookies and, of course, better mood:) Yay!

The rest of the evening was quite simple --- traditional drinks and snacks, drinking games, more drinking games, karaoke, complaints from neighbors, dormitory, pillow fight (yeah, kinda gay), wrestling (still gay), sleep.

The next day was quite a hell. We still had to do some sightseeing, so we had to wake up quite early to get to the breakfast (=rice). After the breakfast we made some weird green buns, filled with bean paste (not bad, actually) and then went to see the local botanical garden. Blah blah blah, not interesting stuff. After that a visit to a typical house, a fish market and finally the ride home. If we weren't stuck in a traffic jam for most of the ride and if the bus had some functioning aircon, I would be more than happy, however, we were stuck and the bus was hot, so I arrived to Seoul tired and grumpy. The never ending taxi ride to the dorm certainly did not help, so I was seriously relieved when I got back to the dorm and could just sit down and do nothing. And that's all, folks.

Crappy post, I know\ldots
	\end{content}
\end{post}
